% The formatting of this CV is based on @davidwhogg's layout.

\documentclass[12pt,letterpaper]{article}

% ----- Packages -----
\usepackage[margin=1in]{geometry}    % Page margins
\usepackage{enumitem}               % Customizable lists
\usepackage{hyperref}               % Hyperlinks in PDF
\usepackage{setspace}               % Line spacing (optional)
\usepackage{fontawesome5}
\usepackage{xcolor}                 % Text color
\usepackage[utf8]{inputenc}
\usepackage[T1]{fontenc}
\usepackage{titlesec}
% ----- Document Start -----

\usepackage{color}
\usepackage{fancyhdr}
\usepackage{hyperref}
\usepackage{ifthen}

% \usepackage[yyyymmdd]{datetime}
% \renewcommand{\dateseparator}{-}

% Link formatting.
\definecolor{numcolor}{rgb}{0.5,0.5,0.5}
\definecolor{linkcolor}{rgb}{0,0,0.4}
\definecolor{niceblue}{HTML}{538acf}
\hypersetup{%
    colorlinks=true,        % false: boxed links; true: colored links
    linkcolor=linkcolor,    % color of internal links
    citecolor=linkcolor,    % color of links to bibliography
    filecolor=linkcolor,    % color of file links
    urlcolor=linkcolor      % color of external links
}

% Text formatting.
\newcommand{\foreign}[1]{\textit{#1}}
\newcommand{\etal}{\foreign{et~al.}}
\newcommand{\project}[1]{\textsl{#1}}
\definecolor{grey}{rgb}{0.5,0.5,0.5}
\newcommand{\deemph}[1]{\textcolor{grey}{\footnotesize{#1}}}

% literature links--use doi if you can
  \newcommand{\doi}[2]{\emph{\href{http://dx.doi.org/#1}{{#2}}}}
  \newcommand{\ads}[2]{\href{http://adsabs.harvard.edu/abs/#1}{{#2}}}
  \newcommand{\isbn}[1]{{\footnotesize(\textsc{isbn:}{#1})}}
  \newcommand{\arxiv}[1]{{\href{http://arxiv.org/abs/#1}{arXiv:{#1}}}}

% Section headings.
\renewcommand\familydefault{\sfdefault}
\usepackage{titlesec}

\titleformat{\subsection}
{\normalfont\sffamily\large\bfseries}
{}{0pt}{}[\vspace{-15pt} \enspace \leaders\hrule height 0.4pt \hfill\kern0pt]

\titleformat{\subsubsection}
{\normalfont\sffamily\bfseries}
{}{0pt}{}

\titlespacing{\subsection}{0pt}{2\parskip}{0pt}
\titlespacing{\subsubsection}{0pt}{\parskip}{0pt}

\newcommand{\cvheading}[1]{%
  \addvspace{1ex}%
  \pagebreak[2]%
  \par\noindent
  % Print the heading text
  \textbf{#1}\enspace
  % Insert a rule that fills all remaining horizontal space
  \leaders\hrule height 0.4pt \hfill\kern0pt
  \nopagebreak
  \vspace{-0.4em}%
}

% Set up the custom unordered list.
\newcounter{refpubnum}
\newcommand{\cvlist}{%
    \rightmargin=0in
    \leftmargin=0.15in
    \topsep=0ex
    \partopsep=0pt
    \itemsep=0.2ex
    \parsep=0pt
    \itemindent=-1.0\leftmargin
    \listparindent=0.0\leftmargin
    \settowidth{\labelsep}{~}
    \usecounter{refpubnum}
}

% Set up the custom bullet list for descriptions.
\newlist{bulletdescription}{itemize}{2}
\setlist[bulletdescription]{noitemsep, topsep=-5pt, parsep=0pt, partopsep=0pt, leftmargin=0.25in}
\setlist[bulletdescription,1]{label=$\bullet$}
\setlist[bulletdescription,2]{label=$\circ$}

% Margins and spaces.
\raggedright
\setlength{\oddsidemargin}{0in}
\setlength{\topmargin}{0in}
\setlength{\headsep}{0.20in}
\setlength{\headheight}{0.25in}
\setlength{\textheight}{9.1in}
\addtolength{\topmargin}{-\headsep}
\addtolength{\topmargin}{-\headheight}
\setlength{\textwidth}{6.50in}
\setlength{\parindent}{0in}
\setlength{\parskip}{1ex}

% Headings and footings.
\renewcommand{\headrulewidth}{0pt}
\pagestyle{fancy}
\lhead{\deemph{Adam Boesky}}
\chead{\deemph{Curriculum Vitae}}
\rhead{\deemph{\thepage}}
\cfoot{\deemph{Last updated: \today}}

% Journal names.
\newcommand{\aj}{AJ}
\newcommand{\apj}{ApJ}
\newcommand{\pasp}{PASP}
\newcommand{\mnras}{MNRAS}


\begin{document}\thispagestyle{empty}\sloppy\sloppypar\raggedbottom

\begin{center}
      {\huge \textbf{Adam Boesky}}\\[0.5em]
      {\footnotesize
            \faPhoneSquare \ \href{tel:+1-917-428-3632}{+1(917)\hspace{0.075cm}428-3632}
            \ | \
            \faEnvelope \ \href{mailto:apboesky@gmail.com}{apboesky@gmail.com}
            \ | \
            \faGithub \href{https://github.com/Adam-Boesky}{/Adam-Boesky}
      }
  \end{center}

\subsection{Education}

\textbf{A.B., Harvard University} {\small\textit{\textcolor{gray}{Cambridge, MA} \hfill \textcolor{niceblue}{Aug 2021 -- Present}}}\\
\quad \textcolor{gray}{\textbf{\textit{Majors:}}} Astrophysics, Mathematics \quad \textcolor{gray}{\textbf{\textit{GPA:}}} 3.91 \\
\quad \textcolor{gray}{\textbf{\textit{Relevant Coursework:}}} General Relativity, Information Theory, Machine Learning for \\
\quad \quad Astrophysicists, Advanced Scientific Computing, Differential Geometry, Dynamical Systems


% \begin{list}{}{\cvlist}
%   \item
%         PhD 2015, Department of Physics, New York University. Advisor: Hogg
%   \item
%         MSc 2010, Department of Physics, Queen's University, Canada. Advisor: Widrow
%   \item
%         BSc 2008, Department of Physics, McGill University, Canada.
% \end{list}

\subsection{Professional \& Research Experience}

\textbf{SpaceX} {\small\textit{\textcolor{gray}{Los Angeles, CA} \hfill \textcolor{niceblue}{Summers 2023, 2024}}}\\
\textit{Guidance, Navigation, \& Control Engineering Intern}
\begin{bulletdescription}
      \item
            Developed physical and empirical models for high-fidelity Monte Carlo simulation of the Falcon rocket
      \item
            Drove operational changes to promote launch vehicle reliability and performance
      \item
            Automated day of launch operations and implemented tools to accelerate launch cadence
      \begin{bulletdescription}
            \item
                  Conducted study to assess the value of mission-specific analyses required for launch
      \end{bulletdescription}
\end{bulletdescription}

\vspace{7pt}

\textbf{Center for Astrophysics | Harvard \& Smithsonian} {\small\textit{\textcolor{gray}{Cambridge, MA}}}\\
\textit{Berger Cosmic Transient Group} {\small \textit{\hfill \textcolor{niceblue}{Jun 2022 -- Present}}}
\begin{bulletdescription}
      \item
            Investigated the evolution and demographics of black holes and neutron stars with population synthesis simulations of massive binary stars
      \item
            For senior thesis: created catalog of long-duration transients with novel source extraction methods; studying the detectability of long-duration transients with LSST and \textit{Roman}
\end{bulletdescription}
\textit{Villar Time-Domain Astronomy Data Lab} {\small \textit{\hfill \textcolor{niceblue}{Aug 2023 -- Present}}}
\begin{bulletdescription}
      \item
            Developed a machine learning pipeline to rapidly classify supernovae using their inferred host galaxy's properties; preliminary website \href{http://astrotimelab.com/_pages/splash.html}{here}
\end{bulletdescription}

\vspace{7pt}

\textbf{Harvard School of Engineering \& Applied Sciences} {\small\textit{\textcolor{gray}{Cambridge, MA}}}\\
\textit{Ramanathan Lab} {\small \textit{\hfill \textcolor{niceblue}{Jan 2023 -- July 2024}}}
\begin{bulletdescription}
      \item
            Developed a biologically-realizable computational architecture for basic intelligence; publication 4
\end{bulletdescription}
\textit{Harvard C3 Policing Research Team} {\small \textit{\hfill \textcolor{niceblue}{June 2020 -- Aug 2022}}}
\begin{bulletdescription}
      \item
            Quantified impact of novel Counter Criminal Continuum (C3) policing method; presented results to Massachusetts government and law enforcement officials
      \item
            Identified statistically significant positive trends in business, crime, education, health, and housing data
\end{bulletdescription}

% \begin{list}{}{\cvlist}
%   \item
%         Research Engineer, Google DeepMind, 2024--present.
%   \item
%         Research Scientist, Flatiron Institute, 2022--2024.
%   \item
%         Associate Research Scientist, Flatiron Institute, 2017--2022.
%   \item
%         Sagan Postdoctoral Fellow, University of Washington, 2015--2017.
% \end{list}

\subsection{Popular open-source software}
\begin{list}{}{\cvlist}
  \item \href{https://github.com/Adam-Boesky/astro_SPLASH}{{\bf astro{\_}SPLASH}} --- (1 stars / 1 forks) \\
Supernova classification Pipeline Leveraging Attributes of Supernova Hosts (SPLASH): A machine learning pipeline that classifies supernovae by first inferring their host's properties 

\item \href{https://github.com/TeamCOMPAS/COMPAS}{{\bf COMPAS}} --- (72 stars / 72 forks) \\
COMPAS rapid binary population synthesis code \href{http://compas.science}{[docs]}

\item \href{https://github.com/Adam-Boesky/long_transients}{{\bf long{\_}transients}} --- (1 stars / 0 forks) \\
Finding long-duration transient candidates 

\item \href{https://github.com/jdinovi/HOOTSim}{{\bf HOOTSim}} --- (2 stars / 0 forks) \\
N-Body Simulator 

\item \href{https://github.com/Adam-Boesky/RL_CMAES}{{\bf RL{\_}CMAES}} --- (0 stars / 0 forks) \\
Implementing the covariance matrix adaptation evolution strategy in a reinforcement learning framework 

\item \href{https://github.com/Adam-Boesky/Exploring_Parameter_Space}{{\bf Exploring{\_}Parameter{\_}Space}} --- (1 stars / 0 forks) \\
Investigating the redshift evolution of the compact object merger rate 
\end{list}

\ifdefined\withpubs
  \subsection{Publications}
  refereed: 2 / first author: 3 / citations: 17 / h-index: 2 (2025-06-04)

  \subsubsection{Refereed publications}
  \begin{list}{}{\cvlist}
    \item[{\color{numcolor}\scriptsize2}] \textbf{Boesky, Adam}; Broekgaarden, Floor S.; \& Berger, Edo, 2024, \doi{10.3847/1538-4357/ad7fe3}{Investigating the Cosmological Rate of Compact Object Mergers from Isolated Massive Binary Stars}, The Astrophysical Journal, \textbf{976}, 24 (\arxiv{2405.01630}) [\href{https://ui.adsabs.harvard.edu/abs/2024ApJ...976...24B}{9 citations}]

\item[{\color{numcolor}\scriptsize1}] \textbf{Boesky, Adam}; Broekgaarden, Floor S.; \& Berger, Edo, 2024, \doi{10.3847/1538-4357/ad7fe4}{The Binary Black Hole Merger Rate Deviates from the Cosmic Star Formation Rate: A Tug of War between Metallicity and Delay Times}, The Astrophysical Journal, \textbf{976}, 23 (\arxiv{2405.01623}) [\href{https://ui.adsabs.harvard.edu/abs/2024ApJ...976...23B}{7 citations}]
  \end{list}

  \subsubsection{Preprints \& white papers}
  \begin{list}{}{\cvlist}
    \item[{\color{numcolor}\scriptsize2}] Schuetz, Ann-Kathrin; Migala, Alexander; \textbf{Boesky, Adam}; Poon, Alan W. P.; \etal, 2025, RESOLVE: Rare Event Surrogate Likelihood for Gravitational Wave Paleontology Parameter Estimation, ArXiv (\arxiv{2506.00757})

\item[{\color{numcolor}\scriptsize1}] \textbf{Boesky, Adam}; Villar, V. Ashley; Gagliano, Alexander; \& Hsu, Brian, 2025, SPLASH: A Rapid Host-Based Supernova Classifier for Wide-Field Time-Domain Surveys, ArXiv (\arxiv{2506.00121})
  \end{list}
\fi

\subsection{Mentorship}

I collaborate with and mentor many students and postdocs, often on a single
project.
Below is a list of the group members who I have formally mentored as part of the
Flatiron Research Fellowship and Pre-doctoral Fellowship at the Center for
Computational Astrophysics.

\begin{list}{}{\cvlist}
\item \emph{Current postdocs:}
  Thavisha Dharmawardena,
  Jiayin Dong,
  Nora Eisner,
  Lionel Garcia,
  Joseph Long.
\item \emph{Current students:}
  Quadry Chance,
  Soichiro Hattori.
\item \emph{Former postdocs:}
  Megan Bedell,
  Trevor David,
  Rodrigo Luger.
\item \emph{Former students:}
  Fran Bartoli\'c,
  Eoin Farrell,
  Alex Gagliano,
  Karl Jaehnig,
  Gautam Nagaraj,
  Pa Chia Thao,
  Nhat Quang Hoang Tran.
\end{list}


\subsection{Selected invited talks \& tutorials}
\begin{list}{}{\cvlist}

  \item \emph{Open software for Astrophysics},
      2023, Invited Plenary, 241st AAS Meeting, Seattle.

  \item \emph{Gaussian Processes for EPRV},
      2022, Invited Tutorial, University of Oxford, UK.

  \item \emph{Methods for scalable probabilistic inference},
      2022, Colloquium, University of Illinois Urbana-Champaign.\\
      2022, Colloquium, UC Berkeley.\\
      2022, Colloquium, University of Oxford, UK.\\
      2021, Invited Talk, Institute for Pure \& Applied Mathematics, UCLA.

  \item \emph{Advanced probabilistic modeling},
      2021, Tutorial, Harley Wood Winter School of Astronomy, Australia.

  \item \emph{Open-source software for probabilistic data analysis in astronomy},
      2021, Seminar, Instituto de Astrof\'isica, Portugal.

  \item \emph{Gaussian processes \& stellar variability},
      2021, Seminar, CARMENES Team Meeting.

  \item \emph{Extending JAX with custom C++ \& CUDA},
        2021, Invited Talk, IRIS-HEP Topical Meeting, CERN.

  \item \emph{Open source software for probabilistic data analysis},
        2020, Invited Talk, OzGrav Early Career Researcher Symposium, Australia.

  \item \emph{The why \& how of exoplanet, a domain-specific PyMC3 extension},
        2020, Contributed Talk, PyMC Con.

  \item \emph{A modular ecosystem for probabilistic data analysis},
        2019, Invited Talk, Open Digital Infrastructure in Astronomy conference,
        Kavli Institute for Theoretical Physics.

  \item \emph{Exoplanet population inference, a tutorial},
        2019, Invited Talk, Exostar19 conference,
        Kavli Institute for Theoretical Physics.

  \item \emph{Astronomy as a testbed for statistical method development},
        2019, Colloquium, Center for Statistics and Machine Learning,
        Princeton.

  \item \emph{Data-driven discovery in the astronomical time domain},
        2018, Colloquium, Institute for Theory and Computation,
        Harvard-Smithsonian Center for Astrophysics.\\
        2018, Colloquium, University of California, Santa Cruz.\\
        2017, Interdisciplinary Colloquium, CIERA, Northwestern University.

  \item \emph{A practical introduction to Gaussian Processes for astronomy},
        2017, Invited Talk, Statistical Challenges in Astrophysics,
        University of New South Wales, Australia.

  \item \emph{Long-period transiting planets \& their population},
        2016, Invited talk, Exoplanets I, Davos. \\
        2016, Invited talk, Statistical Challenges of Modern Astrophysics,
        Carnegie Mellon.\\
        2016, Colloquium, Villanova.

  \item \emph{Scalable Gaussian processes \& the search for transiting
          exoplanets}, 2015, Data Science at the LHC, CERN, Geneva.

  \item \emph{Discovery \& characterization of transiting exoplanets \& their
          population}, 2015, Colloquium, University of Washington.

  \item \emph{Hierarchical inference for exoplanet population inference},
        2015, IAU Symposium, Honolulu.

  \item \emph{Data-driven models}, 2015, Extreme precision radial velocities,
        Yale.

  \item \emph{Population inference from noisy \& incomplete catalogs}, 2015,
        Local Group Astrostatistics, University of Michigan.

  \item \emph{Time series analysis, Gaussian Processes, and the search for
          exo-Earths},
        2014, PyData NYC conference, New York.

  \item \emph{Introduction to Gaussian Processes, probabilistic graphical
          models, and deep learning},
        2014, Astro Hack Week, University of Washington.

  \item \emph{An astronomer's introduction to Gaussian processes},
        2014, Bayesian Computing for Astronomical Data Analysis (Summer school at
        Penn State University).

\end{list}

\subsection{Grants}
\begin{list}{}{\cvlist}
  \item NSF-CDS\&E (PI: Agol)
        \emph{Development of fast, multi-dimensional Gaussian Processes for Exoplanet discovery and beyond},
        \$471,048.00, 2019--2022

  \item
        NSF-AAG (PI: Agol),
        \emph{Collaborative Research: Masses and architectures of (potentially
          habitable) exoplanet systems},
        \$491,950, 2016--2018

  \item
        K2 Guest Observer -- Cycle 3 (PI: Penny),
        \emph{Free-Floating and Bound Planet Mass Measurements with K2: Ground- and
          Space-Based Photometry, Event Detection and Modeling},
        \$84,000, 2016--2017

  \item
        K2 Guest Observer -- Cycle 3 (PI: Hogg),
        \emph{Ultra-precise photometry in crowded fields: A self-calibration
          approach},
        \$100,000, 2016--2017

  \item
        XSEDE (PI: Foreman-Mackey),
        \emph{A systematic search for transiting exoplanets using K2},
        100,000 CPU hours, 2015--2016
\end{list}


\subsection{Honors}
\begin{list}{}{\cvlist}

  \item Kavli Fellow, 2015.
  \item Sagan Postdoctoral Fellowship, 2015--2017.
  \item James Arthur Graduate Fellowship, 2014.
  \item Horizon Fellowship in the Natural \& Physical Sciences, 2012.
  \item Henry M. MacCracken Fellowship, 2010.
  \item NSERC Undergraduate Summer Research Award, 2007.

\end{list}

% \ifdefined\withpubs
%     \newpage
% \fi

\subsection{Professional service \& activities}
\begin{list}{}{\cvlist}
  \item Associate Editor-in-Chief --- Journal of Open Source Software
  \item Active Referee ---
        AAS Journals,
        MNRAS,
        PASP,
        A\&A,
        Journal of Statistical Software,
        Journal on Uncertainty Quantification,
        Journal of Open Source Software
  \item Panelist ---
        NSF, NASA, LSSTC
\end{list}

\end{document}
