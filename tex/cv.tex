% The formatting of this CV is based on @davidwhogg's layout.

\documentclass[12pt,letterpaper]{article}

\usepackage{color}
\usepackage{fancyheadings}
\usepackage{hyperref}
\usepackage{ifthen}

% Date formatting.
\usepackage[yyyymmdd]{datetime}
\renewcommand{\dateseparator}{-}

% Link formatting.
\definecolor{linkcolor}{rgb}{0,0,0.4}
\hypersetup{%
    colorlinks=true,        % false: boxed links; true: colored links
    linkcolor=linkcolor,    % color of internal links
    citecolor=linkcolor,    % color of links to bibliography
    filecolor=linkcolor,    % color of file links
    urlcolor=linkcolor      % color of external links
}

% Text formatting.
\newcommand{\foreign}[1]{\textit{#1}}
\newcommand{\etal}{\foreign{et~al}}
\newcommand{\project}[1]{\textsl{#1}}
\definecolor{grey}{rgb}{0.5,0.5,0.5}
\newcommand{\deemph}[1]{\textcolor{grey}{\footnotesize{#1}}}

% literature links--use doi if you can
  \newcommand{\doi}[2]{\emph{\href{http://dx.doi.org/#1}{{#2}}}}
  \newcommand{\ads}[2]{\href{http://adsabs.harvard.edu/abs/#1}{{#2}}}
  \newcommand{\isbn}[1]{{\footnotesize(\textsc{isbn:}{#1})}}
  \newcommand{\arxiv}[1]{{\href{http://arxiv.org/abs/#1}{arXiv:{#1}}}}

% Section headings.
\newcommand{\cvheading}[1]{\addvspace{1ex}\pagebreak[2]\par\textbf{#1}\nopagebreak\vspace{-0.4em}}

% Set up the custom unordered list.
\newcounter{refpubnum}
\newcommand{\cvlist}{%
    \rightmargin=0in
    \leftmargin=0.15in
    \topsep=0ex
    \partopsep=0pt
    \itemsep=0.2ex
    \parsep=0pt
    \itemindent=-1.0\leftmargin
    \listparindent=0.0\leftmargin
    \settowidth{\labelsep}{~}
    \usecounter{refpubnum}
}

% Margins and spaces.
\raggedright
\setlength{\oddsidemargin}{0in}
\setlength{\topmargin}{0in}
\setlength{\headsep}{0.20in}
\setlength{\headheight}{0.25in}
\setlength{\textheight}{9.00in}
\addtolength{\topmargin}{-\headsep}
\addtolength{\topmargin}{-\headheight}
\setlength{\textwidth}{6.50in}
\setlength{\parindent}{0in}
\setlength{\parskip}{1ex}

% Headings and footings.
\renewcommand{\headrulewidth}{0pt}
\pagestyle{fancy}
\lhead{\deemph{Daniel Foreman-Mackey}}
\chead{\deemph{Curriculum Vitae}}
\rhead{\deemph{\thepage}}
\cfoot{\deemph{Last updated: \today}}

% Journal names.
\newcommand{\aj}{AJ}
\newcommand{\apj}{ApJ}
\newcommand{\pasp}{PASP}


\begin{document}\thispagestyle{empty}\sloppy\sloppypar\raggedbottom

\begin{center}
      {\huge \textbf{Adam Boesky}}\\[0.5em]
      {\footnotesize
            \faPhoneSquare \ \href{tel:+1-917-428-3632}{+1(917)\hspace{0.075cm}428-3632}
            \ | \
            \faEnvelope \ \href{mailto:apboesky@gmail.com}{apboesky@gmail.com}
            \ | \
            \faGithub \href{https://github.com/Adam-Boesky}{/Adam-Boesky}
      }
  \end{center}

\subsection{Education}

\textbf{\textit{Candidate} PhD, Harvard University} {\small\textit{\textcolor{gray}{Cambridge, MA} \hfill \textcolor{niceblue}{Aug 2026 -- Present}}}\\
\quad \textcolor{gray}{\textbf{\textit{Subject:}}} Astrophysics \\

\vspace{7pt}


\textbf{BA, Harvard University} {\small\textit{\textcolor{gray}{Cambridge, MA} \hfill \textcolor{niceblue}{Aug 2021 -- May 2025}}}\\
\quad \textcolor{gray}{\textbf{\textit{Majors:}}} Astrophysics, Mathematics \quad \textcolor{gray}{\textbf{\textit{Honors:}}} \textit{cum laude} \\
\quad \textcolor{gray}{\textbf{\textit{Relevant Coursework:}}} General Relativity, Information Theory, Machine Learning for \\
\quad \quad Astrophysicists, Advanced Scientific Computing, Differential Geometry, Dynamical Systems


% \begin{list}{}{\cvlist}
%   \item
%         PhD 2015, Department of Physics, New York University. Advisor: Hogg
%   \item
%         MSc 2010, Department of Physics, Queen's University, Canada. Advisor: Widrow
%   \item
%         BSc 2008, Department of Physics, McGill University, Canada.
% \end{list}

\subsection{Professional \& Research Experience}

\textbf{SpaceX} {\small\textit{\textcolor{gray}{Los Angeles, CA} \hfill \textcolor{niceblue}{Summers 2023, 2024}}}\\
\textit{Guidance, Navigation, \& Control Engineering Intern}
\begin{bulletdescription}
      \item
            Developed physical and empirical models for high-fidelity Monte Carlo simulation of the Falcon rocket
      \item
            Drove operational changes to promote launch vehicle reliability and performance
      \item
            Automated day of launch operations and implemented tools to accelerate launch cadence
      \begin{bulletdescription}
            \item
                  Conducted study to assess the value of mission-specific analyses required for launch
      \end{bulletdescription}
\end{bulletdescription}

\vspace{7pt}

\textbf{Center for Astrophysics | Harvard \& Smithsonian} {\small\textit{\textcolor{gray}{Cambridge, MA}}}\\
\textit{Berger Cosmic Transient Group} {\small \textit{\hfill \textcolor{niceblue}{Jun 2022 -- Present}}}
\begin{bulletdescription}
      \item
            Investigated the evolution and demographics of black holes and neutron stars with population synthesis simulations of massive binary stars
      \item
            For senior thesis: created catalog of long-duration transients with novel source extraction methods; studying the detectability of long-duration transients with LSST and \textit{Roman}
\end{bulletdescription}
\textit{Villar Time-Domain Astronomy Data Lab} {\small \textit{\hfill \textcolor{niceblue}{Aug 2023 -- Present}}}
\begin{bulletdescription}
      \item
            Developed a machine learning pipeline to rapidly classify supernovae using their inferred host galaxy's properties; preliminary website \href{http://astrotimelab.com/_pages/splash.html}{here}
\end{bulletdescription}

\vspace{7pt}

\textbf{Harvard School of Engineering \& Applied Sciences} {\small\textit{\textcolor{gray}{Cambridge, MA}}}\\
\textit{Ramanathan Lab} {\small \textit{\hfill \textcolor{niceblue}{Jan 2023 -- July 2024}}}
\begin{bulletdescription}
      \item
            Developed a biologically-realizable computational architecture for basic intelligence; publication 4
\end{bulletdescription}
\textit{Harvard C3 Policing Research Team} {\small \textit{\hfill \textcolor{niceblue}{June 2020 -- Aug 2022}}}
\begin{bulletdescription}
      \item
            Quantified impact of novel Counter Criminal Continuum (C3) policing method; presented results to Massachusetts government and law enforcement officials
      \item
            Identified statistically significant positive trends in business, crime, education, health, and housing data
\end{bulletdescription}

% \begin{list}{}{\cvlist}
%   \item
%         Research Engineer, Google DeepMind, 2024--present.
%   \item
%         Research Scientist, Flatiron Institute, 2022--2024.
%   \item
%         Associate Research Scientist, Flatiron Institute, 2017--2022.
%   \item
%         Sagan Postdoctoral Fellow, University of Washington, 2015--2017.
% \end{list}

\subsection{Popular open-source software}
\begin{list}{}{\cvlist}
  \item \href{https://github.com/dfm/emcee}{{\bf emcee}} --- 946 stars / 352 forks \\
The Python ensemble sampling toolkit for affine-invariant MCMC \href{http://emcee.readthedocs.io}{[docs]}

\item \href{https://github.com/dfm/george}{{\bf george}} --- 301 stars / 98 forks \\
Fast and flexible Gaussian Process regression in Python \href{http://george.readthedocs.io}{[docs]}

\item \href{https://github.com/dfm/celerite}{{\bf celerite}} --- 124 stars / 31 forks \\
Scalable 1D Gaussian Processes in C++, Python, and Julia \href{http://celerite.rtfd.io}{[docs]}

\item \href{https://github.com/daft-dev/daft}{{\bf daft}} --- 479 stars / 101 forks \\
Render probabilistic graphical models using matplotlib \href{https://docs.daft-pgm.org}{[docs]}

\item \href{https://github.com/dfm/corner.py}{{\bf corner.py}} --- 243 stars / 157 forks \\
Make some beautiful corner plots \href{http://corner.readthedocs.io}{[docs]}

\item \href{https://github.com/dfm/exoplanet}{{\bf exoplanet}} --- 62 stars / 17 forks \\
Fast {\&} scalable MCMC for all your exoplanet needs!  \href{https://exoplanet.dfm.io}{[docs]}
\end{list}

\ifdefined\withpubs
  \subsection{Publications}
  refereed: 99 / first author: 9 / citations: 16,684 / h-index: 44 (2023-09-21)

  \subsubsection{Refereed publications}
  \begin{list}{}{\cvlist}
    \item[{\color{numcolor}\scriptsize51}] Angus, Ruth; Morton, Timothy D.; \textbf{Foreman-Mackey, Daniel}; van Saders, Jennifer; \etal, 2019, \doi{10.3847/1538-3881/ab3c53}{Toward Precise Stellar Ages: Combining Isochrone Fitting with Empirical Gyrochronology}, \aj, \textbf{158}, 173 (\arxiv{1908.07528}) [\href{https://ui.adsabs.harvard.edu/abs/2019AJ....158..173A}{2 citations}]

\item[{\color{numcolor}\scriptsize50}] \textbf{Foreman-Mackey, Daniel}; Farr, Will; Sinha, Manodeep; Archibald, Anne; \etal, 2019, \doi{10.21105/joss.01864}{emcee v3: A Python ensemble sampling toolkit for affine-invariant MCMC}, The Journal of Open Source Software, \textbf{4}, 1864 (\arxiv{1911.07688})

\item[{\color{numcolor}\scriptsize49}] David, Trevor J.; Petigura, Erik A.; Luger, Rodrigo; \textbf{Foreman-Mackey, Daniel}; \etal, 2019, \doi{10.3847/2041-8213/ab4c99}{Four Newborn Planets Transiting the Young Solar Analog V1298 Tau}, \apj, \textbf{885} (\arxiv{1910.04563})

\item[{\color{numcolor}\scriptsize48}] Bedell, Megan; Hogg, David W.; \textbf{Foreman-Mackey, Daniel}; Montet, Benjamin T.; \& Luger, Rodrigo, 2019, \doi{10.3847/1538-3881/ab40a7}{WOBBLE: A Data-driven Analysis Technique for Time-series Stellar Spectra}, \aj, \textbf{158}, 164 (\arxiv{1901.00503}) [\href{https://ui.adsabs.harvard.edu/abs/2019AJ....158..164B}{7 citations}]

\item[{\color{numcolor}\scriptsize47}] Feinstein, Adina D.; Montet, Benjamin T.; \textbf{Foreman-Mackey, Daniel}; Bedell, Megan E.; \etal, 2019, \doi{10.1088/1538-3873/ab291c}{eleanor: An Open-source Tool for Extracting Light Curves from the TESS Full-frame Images}, \pasp, \textbf{131}, 94502 (\arxiv{1903.09152}) [\href{https://ui.adsabs.harvard.edu/abs/2019PASP..131i4502F}{9 citations}]

\item[{\color{numcolor}\scriptsize46}] Kruse, Ethan; Agol, Eric; Luger, Rodrigo; \& \textbf{Foreman-Mackey, Daniel}, 2019, \doi{10.3847/1538-4365/ab346b}{Detection of Hundreds of New Planet Candidates and Eclipsing Binaries in K2 Campaigns 0-8}, The Astrophysical Journal Supplement Series, \textbf{244}, 11 (\arxiv{1907.10806}) [\href{https://ui.adsabs.harvard.edu/abs/2019ApJS..244...11K}{2 citations}]

\item[{\color{numcolor}\scriptsize45}] Angus, Ruth; Morton, Timothy; \& \textbf{Foreman-Mackey, Daniel}, 2019, \doi{10.21105/joss.01469}{stardate: Combining dating methods for better stellar ages}, The Journal of Open Source Software, \textbf{4}, 1469

\item[{\color{numcolor}\scriptsize44}] Kostov, Veselin B.; Schlieder, Joshua E.; Barclay, Thomas; Quintana, Elisa V.; \etal\ (incl.\ \textbf{DFM}), 2019, \doi{10.3847/1538-3881/ab2459}{The L 98-59 System: Three Transiting, Terrestrial-size Planets Orbiting a Nearby M Dwarf}, \aj, \textbf{158}, 32 (\arxiv{1903.08017}) [\href{https://ui.adsabs.harvard.edu/abs/2019AJ....158...32K}{9 citations}]

\item[{\color{numcolor}\scriptsize43}] Siemiginowska, Aneta; Eadie, Gwendolyn; Czekala, Ian; Feigelson, Eric; \etal\ (incl.\ \textbf{DFM}), 2019, The Next Decade of Astroinformatics and Astrostatistics, Bulletin of the American Astronomical Society, \textbf{51}, 355 (\arxiv{1903.06796})

\item[{\color{numcolor}\scriptsize42}] Van Eylen, Vincent; Albrecht, Simon; Huang, Xu; MacDonald, Mariah G.; \etal\ (incl.\ \textbf{DFM}), 2019, \doi{10.3847/1538-3881/aaf22f}{The Orbital Eccentricity of Small Planet Systems}, \aj, \textbf{157}, 61 (\arxiv{1807.00549}) [\href{https://ui.adsabs.harvard.edu/abs/2019AJ....157...61V}{23 citations}]

\item[{\color{numcolor}\scriptsize41}] Luger, Rodrigo; Agol, Eric; \textbf{Foreman-Mackey, Daniel}; Fleming, David P.; \etal, 2019, \doi{10.3847/1538-3881/aae8e5}{starry: Analytic Occultation Light Curves}, \aj, \textbf{157}, 64 (\arxiv{1810.06559}) [\href{https://ui.adsabs.harvard.edu/abs/2019AJ....157...64L}{18 citations}]

\item[{\color{numcolor}\scriptsize40}] Brewer, John M.; Wang, Songhu; Fischer, Debra A.; \& \textbf{Foreman-Mackey, Daniel}, 2018, \doi{10.3847/2041-8213/aae710}{Compact Multi-planet Systems are more Common around Metal-poor Hosts}, \apj, \textbf{867} (\arxiv{1810.10009}) [\href{https://ui.adsabs.harvard.edu/abs/2018ApJ...867L...3B}{7 citations}]

\item[{\color{numcolor}\scriptsize39}] Ness, Melissa K.; Silva Aguirre, Victor; Lund, Mikkel N.; Cantiello, Matteo; \etal\ (incl.\ \textbf{DFM}), 2018, \doi{10.3847/1538-4357/aadb40}{Inference of Stellar Parameters from Brightness Variations}, \apj, \textbf{866}, 15 (\arxiv{1805.04519}) [\href{https://ui.adsabs.harvard.edu/abs/2018ApJ...866...15N}{2 citations}]

\item[{\color{numcolor}\scriptsize38}] Brewer, Brendon; \& \textbf{Foreman-Mackey, Daniel}, 2018, \doi{10.18637/jss.v086.i07}{DNest4: Diffusive Nested Sampling in C++ and Python}, Journal of Statistical Software, \textbf{86}, 1 (\arxiv{1606.03757}) [\href{https://scholar.google.com/scholar?cites=789224875040810871}{14 citations}]

\item[{\color{numcolor}\scriptsize37}] Luger, Rodrigo; Kruse, Ethan; \textbf{Foreman-Mackey, Daniel}; Agol, Eric; \& Saunders, Nicholas, 2018, \doi{10.3847/1538-3881/aad230}{An Update to the EVEREST K2 Pipeline: Short Cadence, Saturated Stars, and Kepler-like Photometry Down to Kp = 15}, \aj, \textbf{156}, 99 (\arxiv{1702.05488}) [\href{https://ui.adsabs.harvard.edu/abs/2018AJ....156...99L}{54 citations}]

\item[{\color{numcolor}\scriptsize36}] Teague, Richard; \& \textbf{Foreman-Mackey, Daniel}, 2018, \doi{10.3847/2515-5172/aae265}{A Robust Method to Measure Centroids of Spectral Lines}, Research Notes of the American Astronomical Society, \textbf{2}, 173 (\arxiv{1809.10295}) [\href{https://ui.adsabs.harvard.edu/abs/2018RNAAS...2..173T}{6 citations}]

\item[{\color{numcolor}\scriptsize35}] Teague, Richard; Bae, Jaehan; Bergin, Edwin A.; Birnstiel, Tilman; \& \textbf{Foreman-Mackey, Daniel}, 2018, \doi{10.3847/2041-8213/aac6d7}{A Kinematical Detection of Two Embedded Jupiter-mass Planets in HD 163296}, \apj, \textbf{860} (\arxiv{1805.10290}) [\href{https://ui.adsabs.harvard.edu/abs/2018ApJ...860L..12T}{57 citations}]

\item[{\color{numcolor}\scriptsize34}] Hogg, David W.; \& \textbf{Foreman-Mackey, Daniel}, 2018, \doi{10.3847/1538-4365/aab76e}{Data Analysis Recipes: Using Markov Chain Monte Carlo}, The Astrophysical Journal Supplement Series, \textbf{236}, 11 (\arxiv{1710.06068}) [\href{https://ui.adsabs.harvard.edu/abs/2018ApJS..236...11H}{30 citations}]

\item[{\color{numcolor}\scriptsize33}] Angus, Ruth; Morton, Timothy; Aigrain, Suzanne; \textbf{Foreman-Mackey, Daniel}; \& Rajpaul, Vinesh, 2018, \doi{10.1093/mnras/stx2109}{Inferring probabilistic stellar rotation periods using Gaussian processes}, \mnras, \textbf{474}, 2094 (\arxiv{1706.05459}) [\href{https://ui.adsabs.harvard.edu/abs/2018MNRAS.474.2094A}{41 citations}]

\item[{\color{numcolor}\scriptsize32}] \textbf{Foreman-Mackey, Daniel}, 2018, \doi{10.3847/2515-5172/aaaf6c}{Scalable Backpropagation for Gaussian Processes using Celerite}, Research Notes of the American Astronomical Society, \textbf{2}, 31 (\arxiv{1801.10156}) [\href{https://ui.adsabs.harvard.edu/abs/2018RNAAS...2...31F}{6 citations}]

\item[{\color{numcolor}\scriptsize31}] \textbf{Foreman-Mackey, Daniel}; Agol, Eric; Ambikasaran, Sivaram; \& Angus, Ruth, 2017, \doi{10.3847/1538-3881/aa9332}{Fast and Scalable Gaussian Process Modeling with Applications to Astronomical Time Series}, \aj, \textbf{154}, 220 (\arxiv{1703.09710}) [\href{https://ui.adsabs.harvard.edu/abs/2017AJ....154..220F}{99 citations}]

\item[{\color{numcolor}\scriptsize30}] Grunblatt, Samuel K.; Huber, Daniel; Gaidos, Eric; Lopez, Eric D.; \etal\ (incl.\ \textbf{DFM}), 2017, \doi{10.3847/1538-3881/aa932d}{Seeing Double with K2: Testing Re-inflation with Two Remarkably Similar Planets around Red Giant Branch Stars}, \aj, \textbf{154}, 254 (\arxiv{1706.05865}) [\href{https://ui.adsabs.harvard.edu/abs/2017AJ....154..254G}{26 citations}]

\item[{\color{numcolor}\scriptsize29}] Montet, Benjamin T.; Tovar, Guadalupe; \& \textbf{Foreman-Mackey, Daniel}, 2017, \doi{10.3847/1538-4357/aa9e00}{Long-term Photometric Variability in Kepler Full-frame Images: Magnetic Cycles of Sun-like Stars}, \apj, \textbf{851}, 116 (\arxiv{1705.07928}) [\href{https://ui.adsabs.harvard.edu/abs/2017ApJ...851..116M}{32 citations}]

\item[{\color{numcolor}\scriptsize28}] Luger, Rodrigo; \textbf{Foreman-Mackey, Daniel}; \& Hogg, David W., 2017, \doi{10.3847/2515-5172/aa96b5}{Linear Models for Systematics and Nuisances}, Research Notes of the American Astronomical Society, \textbf{1}, 7 (\arxiv{1710.11136}) [\href{https://ui.adsabs.harvard.edu/abs/2017RNAAS...1....7L}{3 citations}]

\item[{\color{numcolor}\scriptsize27}] Price-Whelan, Adrian M.; \& \textbf{Foreman-Mackey, Daniel}, 2017, \doi{10.21105/joss.00357}{schwimmbad: A uniform interface to parallel processing pools in Python}, The Journal of Open Source Software, \textbf{2}, 357 [\href{https://ui.adsabs.harvard.edu/abs/2017JOSS....2..357P}{7 citations}]

\item[{\color{numcolor}\scriptsize26}] Luger, Rodrigo; Sestovic, Marko; Kruse, Ethan; Grimm, Simon L.; \etal\ (incl.\ \textbf{DFM}), 2017, \doi{10.1038/s41550-017-0129}{A seven-planet resonant chain in TRAPPIST-1}, Nature Astronomy, \textbf{1}, 129 (\arxiv{1703.04166}) [\href{https://ui.adsabs.harvard.edu/abs/2017NatAs...1E.129L}{127 citations}]

\item[{\color{numcolor}\scriptsize25}] Price-Whelan, Adrian M.; Hogg, David W.; \textbf{Foreman-Mackey, Daniel}; \& Rix, Hans-Walter, 2017, \doi{10.3847/1538-4357/aa5e50}{The Joker: A Custom Monte Carlo Sampler for Binary-star and Exoplanet Radial Velocity Data}, \apj, \textbf{837}, 20 (\arxiv{1610.07602}) [\href{https://ui.adsabs.harvard.edu/abs/2017ApJ...837...20P}{24 citations}]

\item[{\color{numcolor}\scriptsize24}] \textbf{Foreman-Mackey, Daniel}; Morton, Timothy D.; Hogg, David W.; Agol, Eric; \& Sch{\"o}lkopf, Bernhard, 2016, \doi{10.3847/0004-6256/152/6/206}{The Population of Long-period Transiting Exoplanets}, \aj, \textbf{152}, 206 (\arxiv{1607.08237}) [\href{https://ui.adsabs.harvard.edu/abs/2016AJ....152..206F}{45 citations}]

\item[{\color{numcolor}\scriptsize23}] Henderson, Calen B.; Poleski, Rados{\l}aw; Penny, Matthew; Street, Rachel A.; \etal\ (incl.\ \textbf{DFM}), 2016, \doi{10.1088/1538-3873/128/970/124401}{Campaign 9 of the K2 Mission: Observational Parameters, Scientific Drivers, and Community Involvement for a Simultaneous Space- and Ground-based Microlensing Survey}, \pasp, \textbf{128}, 124401 (\arxiv{1512.09142}) [\href{https://ui.adsabs.harvard.edu/abs/2016PASP..128l4401H}{49 citations}]

\item[{\color{numcolor}\scriptsize22}] Hogg, David W.; Casey, Andrew R.; Ness, Melissa; Rix, Hans-Walter; \etal\ (incl.\ \textbf{DFM}), 2016, \doi{10.3847/1538-4357/833/2/262}{Chemical Tagging Can Work: Identification of Stellar Phase-space Structures Purely by Chemical-abundance Similarity}, \apj, \textbf{833}, 262 (\arxiv{1601.05413}) [\href{https://ui.adsabs.harvard.edu/abs/2016ApJ...833..262H}{45 citations}]

\item[{\color{numcolor}\scriptsize21}] Luger, Rodrigo; Agol, Eric; Kruse, Ethan; Barnes, Rory; \etal\ (incl.\ \textbf{DFM}), 2016, \doi{10.3847/0004-6256/152/4/100}{EVEREST: Pixel Level Decorrelation of K2 Light Curves}, \aj, \textbf{152}, 100 (\arxiv{1607.00524}) [\href{https://ui.adsabs.harvard.edu/abs/2016AJ....152..100L}{112 citations}]

\item[{\color{numcolor}\scriptsize20}] Angus, Ruth; Aigrain, Suzanne; \& \textbf{Foreman-Mackey, Daniel}, 2016, \doi{10.1017/S1743921316002738}{Stellar rotation period inference with Gaussian processes}, IAU Focus Meeting, \textbf{29A}, 191

\item[{\color{numcolor}\scriptsize19}] Wang, Dun; Hogg, David W.; \textbf{Foreman-Mackey, Daniel}; \& Sch{\"o}lkopf, Bernhard, 2016, \doi{10.1088/1538-3873/128/967/094503}{A Causal, Data-driven Approach to Modeling the Kepler Data}, \pasp, \textbf{128}, 94503 (\arxiv{1508.01853}) [\href{https://ui.adsabs.harvard.edu/abs/2016PASP..128i4503W}{9 citations}]

\item[{\color{numcolor}\scriptsize18}] Fischer, Debra A.; Anglada-Escude, Guillem; Arriagada, Pamela; Baluev, Roman V.; \etal\ (incl.\ \textbf{DFM}), 2016, \doi{10.1088/1538-3873/128/964/066001}{State of the Field: Extreme Precision Radial Velocities}, \pasp, \textbf{128}, 66001 (\arxiv{1602.07939}) [\href{https://ui.adsabs.harvard.edu/abs/2016PASP..128f6001F}{117 citations}]

\item[{\color{numcolor}\scriptsize17}] \textbf{Foreman-Mackey, Daniel}, 2016, \doi{10.21105/joss.00024}{corner.py: Scatterplot matrices in Python}, The Journal of Open Source Software, \textbf{1}, 2 [\href{1835087844145558435,17836006976722650130,7443348433327806275,14220488595059618709,12820425635803494730,7284810048757141243,17415935839493019063}{527 citations}]

\item[{\color{numcolor}\scriptsize16}] Sch{\"o}lkopf, Bernhard; Hogg, David W.; Wang, Dun; \textbf{Foreman-Mackey, Daniel}; \etal, 2016, \doi{10.1073/pnas.1511656113}{Modeling confounding by half-sibling regression}, PNAS, \textbf{113}, 27 [\href{https://scholar.google.com/scholar?cites=2429561747341807338}{15 citations}]

\item[{\color{numcolor}\scriptsize15}] Angus, Ruth; \textbf{Foreman-Mackey, Daniel}; \& Johnson, John A., 2016, \doi{10.3847/0004-637X/818/2/109}{Systematics-insensitive Periodic Signal Search with K2}, \apj, \textbf{818}, 109 (\arxiv{1505.07105}) [\href{https://ui.adsabs.harvard.edu/abs/2016ApJ...818..109A}{17 citations}]

\item[{\color{numcolor}\scriptsize14}] Ambikasaran, Sivaram; \textbf{Foreman-Mackey, Daniel}; Greengard, Leslie; Hogg, David W.; \& O'Neil, Michael, 2016, \doi{10.1109/TPAMI.2015.2448083}{Fast Direct Methods for Gaussian Processes}, IEEE Transactions on Pattern Analysis and Machine Intelligence, \textbf{38}, 252 (\arxiv{1403.6015}) [\href{https://scholar.google.com/scholar?cites=4840899390891567426,9641158393712381489}{179 citations}]

\item[{\color{numcolor}\scriptsize13}] Montet, Benjamin T.; Morton, Timothy D.; \textbf{Foreman-Mackey, Daniel}; Johnson, John Asher; \etal, 2015, \doi{10.1088/0004-637X/809/1/25}{Stellar and Planetary Properties of K2 Campaign 1 Candidates and Validation of 17 Planets, Including a Planet Receiving Earth-like Insolation}, \apj, \textbf{809}, 25 (\arxiv{1503.07866}) [\href{https://ui.adsabs.harvard.edu/abs/2015ApJ...809...25M}{70 citations}]

\item[{\color{numcolor}\scriptsize12}] Barclay, Thomas; Quintana, Elisa V.; Adams, Fred C.; Ciardi, David R.; \etal\ (incl.\ \textbf{DFM}), 2015, \doi{10.1088/0004-637X/809/1/7}{The Five Planets in the Kepler-296 Binary System All Orbit the Primary: A Statistical and Analytical Analysis}, \apj, \textbf{809}, 7 (\arxiv{1505.01845}) [\href{https://ui.adsabs.harvard.edu/abs/2015ApJ...809....7B}{23 citations}]

\item[{\color{numcolor}\scriptsize11}] Angus, Ruth; Aigrain, Suzanne; \textbf{Foreman-Mackey, Daniel}; \& McQuillan, Amy, 2015, \doi{10.1093/mnras/stv423}{Calibrating gyrochronology using Kepler asteroseismic targets}, \mnras, \textbf{450}, 1787 (\arxiv{1502.06965}) [\href{https://ui.adsabs.harvard.edu/abs/2015MNRAS.450.1787A}{75 citations}]

\item[{\color{numcolor}\scriptsize10}] \textbf{Foreman-Mackey, Daniel}; Montet, Benjamin T.; Hogg, David W.; Morton, Timothy D.; \etal, 2015, \doi{10.1088/0004-637X/806/2/215}{A Systematic Search for Transiting Planets in the K2 Data}, \apj, \textbf{806}, 215 (\arxiv{1502.04715}) [\href{https://ui.adsabs.harvard.edu/abs/2015ApJ...806..215F}{81 citations}]

\item[{\color{numcolor}\scriptsize9}] Weisz, Daniel R.; Johnson, L. Clifton; \textbf{Foreman-Mackey, Daniel}; Dolphin, Andrew E.; \etal, 2015, \doi{10.1088/0004-637X/806/2/198}{The High-mass Stellar Initial Mass Function in M31 Clusters}, \apj, \textbf{806}, 198 (\arxiv{1502.06621}) [\href{https://ui.adsabs.harvard.edu/abs/2015ApJ...806..198W}{32 citations}]

\item[{\color{numcolor}\scriptsize8}] Sch{\"o}lkopf, Bernhard; Hogg, David W.; Wang, Dun; \textbf{Foreman-Mackey, Daniel}; \etal, 2015, Removing systematic errors for exoplanet search via latent causes, ICML, \textbf{37}, 2218 (\arxiv{1505.03036}) [\href{https://scholar.google.com/scholar?cites=11768165421845046384}{7 citations}]

\item[{\color{numcolor}\scriptsize7}] Barclay, Thomas; Endl, Michael; Huber, Daniel; \textbf{Foreman-Mackey, Daniel}; \etal, 2015, \doi{10.1088/0004-637X/800/1/46}{Radial Velocity Observations and Light Curve Noise Modeling Confirm that Kepler-91b is a Giant Planet Orbiting a Giant Star}, \apj, \textbf{800}, 46 (\arxiv{1408.3149}) [\href{https://ui.adsabs.harvard.edu/abs/2015ApJ...800...46B}{45 citations}]

\item[{\color{numcolor}\scriptsize6}] \textbf{Foreman-Mackey, Daniel}; Hogg, David W.; \& Morton, Timothy D., 2014, \doi{10.1088/0004-637X/795/1/64}{Exoplanet Population Inference and the Abundance of Earth Analogs from Noisy, Incomplete Catalogs}, \apj, \textbf{795}, 64 (\arxiv{1406.3020}) [\href{https://ui.adsabs.harvard.edu/abs/2014ApJ...795...64F}{144 citations}]

\item[{\color{numcolor}\scriptsize5}] Dawson, Rebekah I.; Johnson, John Asher; Fabrycky, Daniel C.; \textbf{Foreman-Mackey, Daniel}; \etal, 2014, \doi{10.1088/0004-637X/791/2/89}{Large Eccentricity, Low Mutual Inclination: The Three-dimensional Architecture of a Hierarchical System of Giant Planets}, \apj, \textbf{791}, 89 (\arxiv{1405.5229}) [\href{https://ui.adsabs.harvard.edu/abs/2014ApJ...791...89D}{47 citations}]

\item[{\color{numcolor}\scriptsize4}] Dorman, Claire E.; Widrow, Lawrence M.; Guhathakurta, Puragra; Seth, Anil C.; \etal\ (incl.\ \textbf{DFM}), 2013, \doi{10.1088/0004-637X/779/2/103}{A New Approach to Detailed Structural Decomposition from the SPLASH and PHAT Surveys: Kicked-up Disk Stars in the Andromeda Galaxy?}, \apj, \textbf{779}, 103 (\arxiv{1310.4179}) [\href{https://ui.adsabs.harvard.edu/abs/2013ApJ...779..103D}{40 citations}]

\item[{\color{numcolor}\scriptsize3}] Brewer, Brendon J.; \textbf{Foreman-Mackey, Daniel}; \& Hogg, David W., 2013, \doi{10.1088/0004-6256/146/1/7}{Probabilistic Catalogs for Crowded Stellar Fields}, \aj, \textbf{146}, 7 (\arxiv{1211.5805}) [\href{https://ui.adsabs.harvard.edu/abs/2013AJ....146....7B}{28 citations}]

\item[{\color{numcolor}\scriptsize2}] \textbf{Foreman-Mackey, Daniel}; Hogg, David W.; Lang, Dustin; \& Goodman, Jonathan, 2013, \doi{10.1086/670067}{emcee: The MCMC Hammer}, \pasp, \textbf{125}, 306 (\arxiv{1202.3665}) [\href{https://ui.adsabs.harvard.edu/abs/2013PASP..125..306F}{3128 citations}]

\item[{\color{numcolor}\scriptsize1}] Weisz, Daniel R.; Fouesneau, Morgan; Hogg, David W.; Rix, Hans-Walter; \etal\ (incl.\ \textbf{DFM}), 2013, \doi{10.1088/0004-637X/762/2/123}{The Panchromatic Hubble Andromeda Treasury. IV. A Probabilistic Approach to Inferring the High-mass Stellar Initial Mass Function and Other Power-law Functions}, \apj, \textbf{762}, 123 (\arxiv{1211.6105}) [\href{https://ui.adsabs.harvard.edu/abs/2013ApJ...762..123W}{29 citations}]
  \end{list}

  \subsubsection{Preprints \& white papers}
  \begin{list}{}{\cvlist}
    \item[{\color{numcolor}\scriptsize15}] Blunt, Sarah; Carvalho, Adolfo; David, Trevor J.; Beichman, Charles; \etal\ (incl.\ \textbf{DFM}), 2023, Overfitting Affects the Reliability of Radial Velocity Mass Estimates of the V1298 Tau Planets, ArXiv (\arxiv{2306.08145})

\item[{\color{numcolor}\scriptsize14}] Gagliano, Alexander; Contardo, Gabriella; \textbf{Foreman-Mackey, Daniel}; Malz, Alex I.; \& Aleo, Patrick D., 2023, \doi{10.48550/arXiv.2305.08894}{First Impressions: Early-Time Classification of Supernovae using Host Galaxy Information and Shallow Learning}, ArXiv (\arxiv{2305.08894})

\item[{\color{numcolor}\scriptsize13}] Tran, Quang H.; Bedell, Megan; \textbf{Foreman-Mackey, Daniel}; \& Luger, Rodrigo, 2023, \doi{10.48550/arXiv.2305.00988}{Joint Modeling of Radial Velocities and Photometry with a Gaussian Process Framework}, ArXiv (\arxiv{2305.00988})

\item[{\color{numcolor}\scriptsize12}] Dong, Jiayin; \& \textbf{Foreman-Mackey, Daniel}, 2023, \doi{10.48550/arXiv.2305.14220}{A Hierarchical Bayesian Framework for Inferring the Stellar Obliquity Distribution}, ArXiv (\arxiv{2305.14220})

\item[{\color{numcolor}\scriptsize11}] Eadie, Gwendolyn M.; Speagle, Joshua S.; Cisewski-Kehe, Jessi; \textbf{Foreman-Mackey, Daniel}; \etal, 2023, \doi{10.48550/arXiv.2302.04703}{Practical Guidance for Bayesian Inference in Astronomy}, ArXiv (\arxiv{2302.04703})

\item[{\color{numcolor}\scriptsize10}] Yahalomi, Daniel A.; Angus, Ruth; Spergel, David N.; \& \textbf{Foreman-Mackey, Daniel}, 2023, \doi{10.48550/arXiv.2302.05064}{Detecting Solar System Analogs through Joint Radial Velocity/Astrometric Surveys}, ArXiv (\arxiv{2302.05064})

\item[{\color{numcolor}\scriptsize9}] Edwards, Thomas D. P.; Wong, Kaze W. K.; Lam, Kelvin K. H.; Coogan, Adam; \etal\ (incl.\ \textbf{DFM}), 2023, \doi{10.48550/arXiv.2302.05329}{ripple: Differentiable and Hardware-Accelerated Waveforms for Gravitational Wave Data Analysis}, ArXiv (\arxiv{2302.05329}) [\href{https://ui.adsabs.harvard.edu/abs/2023arXiv230205329E}{2 citations}]

\item[{\color{numcolor}\scriptsize8}] Wong, Kaze W. K.; Gabri{\'e}, Marylou; \& \textbf{Foreman-Mackey, Daniel}, 2022, \doi{10.48550/arXiv.2211.06397}{flowMC: Normalizing-flow enhanced sampling package for probabilistic inference in Jax}, ArXiv (\arxiv{2211.06397}) [\href{https://ui.adsabs.harvard.edu/abs/2022arXiv221106397W}{4 citations}]

\item[{\color{numcolor}\scriptsize7}] Aigrain, Suzanne; \& \textbf{Foreman-Mackey, Daniel}, 2022, \doi{10.48550/arXiv.2209.08940}{Gaussian Process regression for astronomical time-series}, ArXiv (\arxiv{2209.08940}) [\href{https://ui.adsabs.harvard.edu/abs/2022arXiv220908940A}{10 citations}]

\item[{\color{numcolor}\scriptsize6}] Chance, Quadry; \textbf{Foreman-Mackey, Daniel}; Ballard, Sarah; Casey, Andrew; \etal, 2022, \doi{10.48550/arXiv.2206.11275}{paired: A Statistical Framework for Determining Stellar Binarity with Gaia RVs. I. Planet Hosting Binaries}, ArXiv (\arxiv{2206.11275})

\item[{\color{numcolor}\scriptsize5}] Luger, Rodrigo; Bedell, Megan; \textbf{Foreman-Mackey, Daniel}; Crossfield, Ian J. M.; \etal, 2021, \doi{10.48550/arXiv.2110.06271}{Mapping stellar surfaces III: An Efficient, Scalable, and Open-Source Doppler Imaging Model}, ArXiv (\arxiv{2110.06271}) [\href{https://ui.adsabs.harvard.edu/abs/2021arXiv211006271L}{21 citations}]

\item[{\color{numcolor}\scriptsize4}] Wang, Dun; Hogg, David W.; \textbf{Foreman-Mackey, Daniel}; \& Sch{\"o}lkopf, Bernhard, 2017, \doi{10.48550/arXiv.1710.02428}{A pixel-level model for event discovery in time-domain imaging}, ArXiv (\arxiv{1710.02428}) [\href{https://ui.adsabs.harvard.edu/abs/2017arXiv171002428W}{11 citations}]

\item[{\color{numcolor}\scriptsize3}] Barnes, Rory; Deitrick, Russell; Luger, Rodrigo; Driscoll, Peter E.; \etal\ (incl.\ \textbf{DFM}), 2016, \doi{10.48550/arXiv.1608.06919}{The Habitability of Proxima Centauri b I: Evolutionary Scenarios}, ArXiv (\arxiv{1608.06919}) [\href{https://ui.adsabs.harvard.edu/abs/2016arXiv160806919B}{59 citations}]

\item[{\color{numcolor}\scriptsize2}] Montet, Benjamin T.; Angus, Ruth; Barclay, Tom; Dawson, Rebekah; \etal\ (incl.\ \textbf{DFM}), 2013, \doi{10.48550/arXiv.1309.0654}{Maximizing Kepler science return per telemetered pixel: Searching the habitable zones of the brightest stars}, ArXiv (\arxiv{1309.0654})

\item[{\color{numcolor}\scriptsize1}] Hogg, David W.; Angus, Ruth; Barclay, Tom; Dawson, Rebekah; \etal\ (incl.\ \textbf{DFM}), 2013, \doi{10.48550/arXiv.1309.0653}{Maximizing Kepler science return per telemetered pixel: Detailed models of the focal plane in the two-wheel era}, ArXiv (\arxiv{1309.0653})
  \end{list}
\fi

\subsection{Mentorship}

I collaborate with and mentor many students and postdocs, often on a single
project.
Below is a list of the group members who I have formally mentored as part of the
Flatiron Research Fellowship and Pre-doctoral Fellowship at the Center for
Computational Astrophysics.

\begin{list}{}{\cvlist}
\item \emph{Current postdocs:}
  Thavisha Dharmawardena,
  Jiayin Dong,
  Nora Eisner,
  Lionel Garcia,
  Joseph Long.
\item \emph{Current students:}
  Quadry Chance,
  Soichiro Hattori.
\item \emph{Former postdocs:}
  Megan Bedell,
  Trevor David,
  Rodrigo Luger.
\item \emph{Former students:}
  Fran Bartoli\'c,
  Eoin Farrell,
  Alex Gagliano,
  Karl Jaehnig,
  Gautam Nagaraj,
  Pa Chia Thao,
  Nhat Quang Hoang Tran.
\end{list}


\subsection{Selected invited talks \& tutorials}
\begin{list}{}{\cvlist}

  \item \emph{Open software for Astrophysics},
      2023, Invited Plenary, 241st AAS Meeting, Seattle.

  \item \emph{Gaussian Processes for EPRV},
      2022, Invited Tutorial, University of Oxford, UK.

  \item \emph{Methods for scalable probabilistic inference},
      2022, Colloquium, University of Illinois Urbana-Champaign.\\
      2022, Colloquium, UC Berkeley.\\
      2022, Colloquium, University of Oxford, UK.\\
      2021, Invited Talk, Institute for Pure \& Applied Mathematics, UCLA.

  \item \emph{Advanced probabilistic modeling},
      2021, Tutorial, Harley Wood Winter School of Astronomy, Australia.

  \item \emph{Open-source software for probabilistic data analysis in astronomy},
      2021, Seminar, Instituto de Astrof\'isica, Portugal.

  \item \emph{Gaussian processes \& stellar variability},
      2021, Seminar, CARMENES Team Meeting.

  \item \emph{Extending JAX with custom C++ \& CUDA},
        2021, Invited Talk, IRIS-HEP Topical Meeting, CERN.

  \item \emph{Open source software for probabilistic data analysis},
        2020, Invited Talk, OzGrav Early Career Researcher Symposium, Australia.

  \item \emph{The why \& how of exoplanet, a domain-specific PyMC3 extension},
        2020, Contributed Talk, PyMC Con.

  \item \emph{A modular ecosystem for probabilistic data analysis},
        2019, Invited Talk, Open Digital Infrastructure in Astronomy conference,
        Kavli Institute for Theoretical Physics.

  \item \emph{Exoplanet population inference, a tutorial},
        2019, Invited Talk, Exostar19 conference,
        Kavli Institute for Theoretical Physics.

  \item \emph{Astronomy as a testbed for statistical method development},
        2019, Colloquium, Center for Statistics and Machine Learning,
        Princeton.

  \item \emph{Data-driven discovery in the astronomical time domain},
        2018, Colloquium, Institute for Theory and Computation,
        Harvard-Smithsonian Center for Astrophysics.\\
        2018, Colloquium, University of California, Santa Cruz.\\
        2017, Interdisciplinary Colloquium, CIERA, Northwestern University.

  \item \emph{A practical introduction to Gaussian Processes for astronomy},
        2017, Invited Talk, Statistical Challenges in Astrophysics,
        University of New South Wales, Australia.

  \item \emph{Long-period transiting planets \& their population},
        2016, Invited talk, Exoplanets I, Davos. \\
        2016, Invited talk, Statistical Challenges of Modern Astrophysics,
        Carnegie Mellon.\\
        2016, Colloquium, Villanova.

  \item \emph{Scalable Gaussian processes \& the search for transiting
          exoplanets}, 2015, Data Science at the LHC, CERN, Geneva.

  \item \emph{Discovery \& characterization of transiting exoplanets \& their
          population}, 2015, Colloquium, University of Washington.

  \item \emph{Hierarchical inference for exoplanet population inference},
        2015, IAU Symposium, Honolulu.

  \item \emph{Data-driven models}, 2015, Extreme precision radial velocities,
        Yale.

  \item \emph{Population inference from noisy \& incomplete catalogs}, 2015,
        Local Group Astrostatistics, University of Michigan.

  \item \emph{Time series analysis, Gaussian Processes, and the search for
          exo-Earths},
        2014, PyData NYC conference, New York.

  \item \emph{Introduction to Gaussian Processes, probabilistic graphical
          models, and deep learning},
        2014, Astro Hack Week, University of Washington.

  \item \emph{An astronomer's introduction to Gaussian processes},
        2014, Bayesian Computing for Astronomical Data Analysis (Summer school at
        Penn State University).

\end{list}


\subsection{Honors, Awards, and Fellowships}
\begin{list}{}{\cvlist}

  \item \textit{Honorable Mention} NSF Graduate Research Funding Program, 2025
  \item \textit{Finalist} Hertz Fellowship, 2025
  \item \textit{Finalist} Rhodes Scholarship, 2025
  \item \textit{Finalist} Marshall Scholarship, 2025
  \item Harvard Program for Research in Science and Engineering Fellow (\$5000) Summer 2022
  \item Harvard College Research Program Fellow (\$1000), 2022

\end{list}

% \ifdefined\withpubs
%     \newpage
% \fi

\subsection{Professional service \& activities}
\begin{list}{}{\cvlist}
  \item Associate Editor-in-Chief --- Journal of Open Source Software
  \item Active Referee ---
        AAS Journals,
        MNRAS,
        PASP,
        A\&A,
        Journal of Statistical Software,
        Journal on Uncertainty Quantification,
        Journal of Open Source Software
  \item Panelist ---
        NSF, NASA, LSSTC
\end{list}

\end{document}
